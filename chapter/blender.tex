\chapter{Blender}

blender could convert the movie, such as format like '.mp4', to pictures squence, like '.png'.

\begin{itemize}
\item Blender does not support the gif format.
\item For a list of supported formats see \url{https://www.blender.org/manual/data_system/files/media/image_formats.html#image-formats}
\end{itemize}


Two way to choose the object

\begin{itemize}
\item  bpy.data.objects["Cube"]
\item bpy.context.object \footnote{While it’s useful to be able to access data directly by name or as a list, it’s more common to op‐ erate on the user’s selection. The context is always available from bpy.context and can be used to get the active object, scene, tool settings along with many other attributes.

Note that the context is read-only. These values cannot be modified directly, though they may be changed by running API functions or by using the data API.
So bpy.context.object = obj will raise an error.
But bpy.context.scene.objects.active = obj will work as expected.
}
\end{itemize}

Animation

There are 2 ways to add keyframes through Python.
The first is through key properties directly, which is similar to inserting a keyframe from the but‐ ton as a user. You can also manually create the curves and keyframe data, then set the path to the property. Here are examples of both methods.
Both examples insert a keyframe on the active object’s Z axis.


Simple example:

\begin{verbatim}
Using Low-Level Functions:
obj = bpy.context.object
obj.location[2] = 0.0
obj.keyframe_insert(data_path="location", frame=10.0, index=2)
obj.location[2] = 1.0
obj.keyframe_insert(data_path="location", frame=20.0, index=2)
\end{verbatim}



Using Low-Level Functions:

\begin{verbatim}
obj = bpy.context.object
obj.animation_data_create()
obj.animation_data.action = bpy.data.actions.new(name="MyAction")
fcu_z = obj.animation_data.action.fcurves.new(data_path="location", index=2)
fcu_z.keyframe_points.add(2)
fcu_z.keyframe_points[0].co = 10.0, 0.0
fcu_z.keyframe_points[1].co = 20.0, 1.0
\end{verbatim}


\section{The python file for moving object}
\begin{verbatim}
import bpy  
from math import sin
from mathutils import Vector

# useful shortcut
scene = bpy.context.scene

# this shows you all objects in scene
scene.objects.keys()

# when you start default Blender project, first object in scene is a Cube
kostka = scene.objects[0]

# you can change location of object simply by setting the values
kostka.location = (1,2,0)

# same with rotation
kostka.rotation_euler = (45,0,0)

# this will make object cease from current scene
scene.objects.unlink(kostka)

# clear everything for now
scene.camera = None  
for obj in scene.objects:  
    scene.objects.unlink(obj)

# create sphere and make it smooth
bpy.ops.mesh.primitive_uv_sphere_add(location = (2,1,2), size=0.5)  
bpy.ops.object.shade_smooth()  
kule = bpy.context.object

# create new cube
bpy.ops.mesh.primitive_cube_add(location = (-2,1,2))  
kostka = bpy.context.object

# create plane 
bpy.ops.mesh.primitive_plane_add(location=(0,0,0))  
plane = bpy.context.object  
plane.dimensions = (20,20,0)

# for every object add material - here represented just as color
for col, ob in zip([(1, 0, 0), (0,1,0), (0,0,1)], [kule, kostka, plane]):  
    mat = bpy.data.materials.new("mat_" + str(ob.name))
    mat.diffuse_color = col
    ob.data.materials.append(mat)

# now add some light
lamp_data = bpy.data.lamps.new(name="lampa", type='POINT')  
lamp_object = bpy.data.objects.new(name="Lampicka", object_data=lamp_data)  
scene.objects.link(lamp_object)  
lamp_object.location = (0, 0, 12)

# and now set the camera
cam_data = bpy.data.cameras.new(name="cam")  
cam_ob = bpy.data.objects.new(name="Kamerka", object_data=cam_data)  
scene.objects.link(cam_ob)  
cam_ob.location = (-1, -1, 5)  
cam_ob.rotation_euler = (3.14/6,0,-0.3)  
cam = bpy.data.cameras[cam_data.name]  
cam.lens = 10


### animation
# positions = ((0,0,2),(0,1,2),(3,2,1),(3,4,1),(1,2,1))

positions = []
angles = []

end = 500

for i in range(0,end):
    positions.append((i/50., 3*sin(i/5.), 1)) 
    #positions.append((0,0,i))


# start with frame 0
number_of_frame = 0  
bpy.context.scene.frame_end = end

for pozice in positions:

    # now we will describe frame with number $cislo_snimku
    scene.frame_set(number_of_frame)

    # set new location for sphere $kule and new rotation for cube $kostka
    kule.location = pozice
    kule.keyframe_insert(data_path="location", index=-1)


    #kostka.location = pozice
    #kostka.keyframe_insert(data_path="location", index=-1)

    kostka.rotation_euler = pozice
    kostka.keyframe_insert(data_path="rotation_euler", index=-1)

    # move next 10 frames forward - Blender will figure out what to do between this time
    number_of_frame += 1
\end{verbatim}
 

