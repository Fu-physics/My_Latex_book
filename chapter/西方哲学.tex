\chapter{西方哲学}

\section{第一时期:自然秩序}

最早从哲学分离出来的,应该是医学。

毕德格拉斯区分了三中不同生活,他说,来到奥林匹克赛会的有三种人
\begin{itemize}
\item 最低级的是做买卖的人,他们为牟利而来
\item 参加比赛的人,他们为荣誉而来
\item  最好的是观众,他们对正在发生的事情加以思考 \footnote{"观看"与希腊词“理论”是一个意思}
\end{itemize}

他们认为音乐对某些神经错乱颇有疗效,且发现了把弦截成一半,会得到一个高八度的音,“这使得他们认为整个世界就是一个音节,一个数目” \footnote{亚里士多德说的}。

毕德格拉斯定理就是中国的勾股定理。

巴门尼德明确的强调,事物的现象并没有向我们展示实在的构成物质,此思想在柏拉图的哲学中产生了决定性的作用。柏拉图提出真理的理智世界和意见的可见世界之间的区别。

芝诺作为巴门尼德的学生,举例说明了这一点:一粒米种子扔在地上是不会发出声响的,但是如果我们把半蒲式耳种子倒在地上,就会有声音了。芝诺由此得出,我们的感官欺骗的了我们,因此想要到达事物的真理,思想之路要比感觉之路更可靠。

NOTE:


人类最直观,最迫切的一个问题的,我想应该是死亡。正因为我们对死亡的畏惧以及无法解释才使得最开始的祭祀由社会的最高权力机构实施(权力者应该是智力上等人者)。为何哲学在最开始的时候没有涉及到这个问题,这是我很疑惑。思考自热秩序的前提是我们能够将人与自然划一条比较清晰的界限,只有这样才能避开自己,更多的谈及自然。那么也就是说,其实哲学在一开始就在研究人本身。

或许哲学家将这部分留给了巫师,不过这个答案不是很令人满意。


\section{智者派和苏格拉底}

第一批哲学家关注的是自然,而智者派和苏格拉底则讲哲学的问题转到了对人类基本伦理问题的研究\footnote{现在的情况正好相反,人类伦理问题止步不前反倒是自然哲学大步前进}。这一转向可以在下述事实中得到部分解释:前哲学家之间没有能达到任何一种统一的宇宙概念,哲学家在此止步不前。同时此种争论导致了一种怀疑主义的倾向:人类理性是否有能力发现宇宙真理?“我们有没有可能发现普遍真理?”成为了一个新的方向。\footnote{《西方哲学》第二章导言}

三个异乡来的三个智者普罗泰戈拉、高尔基亚、塞拉西马柯(他们自己给自己加上“智者”),他们对雅典娜的人的思想和习俗进行了一番新的审视,提出了一些追根究底的问题,是的雅典娜人不得不考虑自己观念和习俗是基于真理还是仅仅基于惯常的行为习惯。智者派还推动了希腊社会由贵族政体向民主政体的转变。他们把真理看作是一种相对的东西,因此他们利用巧言善辩混淆对错,把不正义的事情说的好像公平合理。将青年从好端端的家庭带走,引导他们去从事要摧毁传统宗教与伦理观点的批判分析。他们的形象已经不同于早期哲学家那种不带任何经济考虑而从事哲学的公正无私的思想家形象。苏格拉底曾在智者门下学习,可是因为穷,他只上的起他们提供的“短期课程”\footnote{让他们描述的有一点儿像如今那些受过高等教育但是专门利用普通民众的‘紧致’利己主义者,例如某些政治家、律师、商人 ... }。

普罗泰戈拉名句:“人是万物的尺度,是存在者的尺度,也是不存在者的尺度。”。 他的相对主义严重打击了人们对有可能大仙真知的信心,也招致了苏格拉底和柏拉图的严厉批评。










































