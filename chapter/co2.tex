\chapter{CO2}

\begin{itemize}

\item many papers talk about the line intensity measurements around 1.6 μm

\end{itemize}



The vibrational term values $G(v)$,[note 3] for an anharmonic oscillator are given, to a first approximation, by

\[{\displaystyle G(v)=\omega _{e}\left(v+{1 \over 2}\right)-\omega _{e}\chi _{e}\left(v+{1 \over 2}\right)^{2}\,}\]
where v is a vibrational quantum number, ωe is the harmonic wavenumber and χe is an anharmonicity constant.

When the molecule is in the gas phase, it can rotate about an axis, perpendicular to the molecular axis, passing through the centre of mass of the molecule. The rotational energy is also quantized, with term values to a first approximation given by
\[F_{v}(J)=B_{v}J\left(J+1\right)-DJ^{2}\left(J+1\right)^{2}\]

where J is a rotational quantum number and D is a centrifugal distortion constant. The rotational constant, Bv depends on the moment of inertia of the molecule, Iv, which varies with the vibrational quantum number, v
\[B_{v}={h \over {8\pi ^{2}cI_{v}}};\quad I_{v}={\frac  {m_{A}m_{B}}{m_{A}+m_{B}}}d_{v}^{2}\]

where mA and mB are the masses of the atoms A and B, and d represents the distance between the atoms. The term values of the ro-vibrational states are found (in the Born–Oppenheimer approximation) by combining the expressions for vibration and rotation.

\[{\displaystyle G(v)+F_{v}(J)=\left[\omega _{e}\left(v+{1 \over 2}\right)+B_{v}J(J+1)\right]-\left[\omega _{e}\chi _{e}\left(v+{1 \over 2}\right)^{2}+DJ^{2}(J+1)^{2}\right]}\]

The first two terms in this expression correspond to a harmonic oscillator and a rigid rotor, the second pair of terms make a correction for anharmonicity and centrifugal distortion. A more general expression was given by Dunham.

The selection rule for electric dipole allowed ro-vibrational transitions, in the case of a diamagnetic diatomic molecule is etc.
$ \Delta v=\pm 1\ (\pm 2,\pm 3,{\textit {etc.}},\Delta J=\pm 1$
The transition with Δv=±1 is known as the fundamental transition. The selection rule has two consequences.

\begin{itemize}
\item Both the vibrational and rotational quantum numbers must change. The transition : $\Delta v=\pm 1,\Delta J=0$ (Q-branch) is forbidden

\item The energy change of rotation can be either subtracted from or added to the energy change of vibration, giving the P- and R- branches of the spectrum, respectively.
\end{itemize}


\subsection{For $CO_{2}$}
\begin{enumerate}
\item transition $\left(0001\right)-\left(10001\right),960.9cm^{-1},4.26\mu m,70.42Thz,\left\langle \wp_{ab}\right\rangle ^{2}=15*10^{-4}D^{2},$
\item transition $\left(0001\right)-\left(10002\right),1063.7cm^{-1},9.4\mu m,31.89Thz,\left\langle \wp_{ab}\right\rangle ^{2}=12*10^{-4}D^{2},$
\item transition $\left(0001\right)-\left(0000\right),2349.1cm^{-1},9.4\mu m,31.89Thz,\left\langle \wp_{ab}\right\rangle ^{2}=0.103D^{2},$
\end{enumerate}
Transition between $\left(0001\right)-\left(10001\right),\left\langle \wp_{ab}\right\rangle ^{2}=15*10^{-4}D^{2}$
to relized 10Mhz

\begin{align*}
I & =\frac{5.4}{15}*10^{4}W/cm^{2}\\
 & =3.6*10^{3}W/cm^{2}
\end{align*}

note: for Rubidium 87

D2$\left(5^{2}S_{1/2}\rightarrow5^{2}P_{3/2}\right)$Transition Dipole
Matrix Element 

$4.22752(87)ea_{0}=3.58424(74)\times10^{-29}C\cdot m=10.8D$

To relized 10Mhz, need $5.4\frac{W}{cm^{2}}$

\subsection{WAVENUMBER/WAVELENGTH CONVERTER}

\begin{align*}
xnm & =10,000,000/xcm^{-1}\\
ycm^{-1} & =10,000,000/ynm
\end{align*}

